\chapter{Implementace ekosystému pro implementaci operací} \label{chap:Implementace}
\section{Požadavky na implementaci}
Vytvořit jednoduché prostředí, nad kterým bude možné zpracovávat operace nad konečnými automaty.

Prostředí musí splňovat následující požadavky:
\begin{enumerate}
    \item Musí umět zpracovávat konečné automaty.
    \begin{enumerate}
        \item Přijmutí konečného automatu
        \item Vypsání konečného automatu
        \item Aplikování základních operací nad konečnými automaty neměnící přijímaný jazyk. (odstranění nekoncových stavů, nedeterminismu a pod.)
    \end{enumerate}
    \item Musí být možné tyto automaty rozšířit/přidat nový typ automatů
    \item Musí být možné vytvářet operace pro práci nad těmito automaty. (Implementace operací)
    \item Musí být možné přidávat nové typy automatů.
    \item Musí být možné na toto prostředí navázat nebo ho používat i v jiných pracích.
\end{enumerate}

\subsection{Rozšíření}
	Přidat zpracování zásobníkového automatu, na němž se ukáže rozšiřitelnost ekosystému o další automaty.

\section{Jazyk a balíčky implementace}
Jako jazyk implementace byl zvolený jazyk JavaScript. Tento jazyk byl zvolen z důvodu, že je podporován obrovskou internetovou komunitou a především knihovna(prostředí) napsané v tomto jazyce půjde použít nejen na serverech a prohlížečích, ale také i na mobilních a desktopových zařízeních.

Pro zpřehlednění čitelnosti kódu je použita verze EcmaScript 2015 [todo teference] a výše a pro typovou kontrolu rozšíření Flow [todo reference] od Facebooku [todo reference]. 

\subsection{Balíčky použité pro běh (produkční prostředí)}
\begin{itemize}
    \item Lodash \cite{lodash}
\end{itemize}

\subsection{Balíčky použité pro vývoj}
\begin{itemize}
    \item  Flow \cite{Flow}
        \begin{itemize}
            \item flow-bin \cite{flow-bin}: spustitelný program, použitý pro statickou typovou kontrolu projektu
        \end{itemize}
    \item Babel \cite{Babel}
        \begin{itemize}
            \item babel-cli \cite{babel-cli}: spustitelný program použitý pro  překlad zdrojových kódů v v novější verzi jazyka JavaScript do starší verze jazyka.
            \item babel-register-cli \cite{babel-register-cli}: překládá jazyk obdobně jako babel-cli, avšak až za běhu programu. Dalo by se říci že rozšiřuje interpret programu. 
            \item babel-plugin-transform-object-rest-spread \cite{babel-plugin-transform-object-rest-spread}: babel balíček pro rozšíření syntaxe jazyka
            \item babel-preset-es2015 \cite{babel-preset-es2015}: babel balíček pro překlad z EcmaScript 2015 do starších verzí.
            \item babel-preset-flow \cite{babel-preset-flow}: babel balíček rozšiřující syntaxi o datové typy pro Flow
        \end{itemize}
\end{itemize}

Zde by mohla vzniknout otázka, proč používat Flow a Babel?

Flow je v implementaci použito jak je výše zmíněno pro typovou kontrolu. Není nijak Nutné používat flow v dalších rozšířeních projektu, avšak věřím že typovost dodá programu více spolehlivosti a také výrazně urychluje vývoj už tím, že zamezuje "runtime" chybám vzniklým právě při nekompatibilních typech u netypovaných jazyků. Výrazným soupeřem Flow je TypeScript, který je stejně vhodný. Zde výběr Flow jako typové nadstavby byl tvolen především pro svou kompatibilitu s Reactem[todo reference], což může zjednodušit případný vývoj grafického prostředí v budoucí projektech.

Babel je v implementaci použit tak jak definují balíčky ve výše uvedeném seznamu. Nepoužití Babel by zamezovalo použití Flow a kód napsaný v EcmaScript 2015 by byl omezený pouze na nejnovější verze interpretů a především kdokoliv kdo by chtěl na tomto kódu stavět by musel být obeznámen s EcmaScript 2015, kdežto takto stačí pouze znalost tradičního (slangově oldschool) JavaScriptu, nebo naopak je možné použít zcela jiný přístup, například beztřídní.

\textbf{Upozornění:} Ke konci dubna 2018 přešel Babel na novouverzi 7.0.0-beta.46. Vzhledem k faktu, že se nejedná o plné uvolnění Babel 7 a že tato práce je psána před vypuštěním Babel 7, jsou výše uvedené Babel knihovny závislé na verzi 6.*. Tento fakt nemá sebemenší vliv na implementaci, ani případné používání praktické části této práce v jiných projektech. Pouze je potřeba aby čtenář věděl, že výše uvedené knihovny jsou nyní "depracated" a tedy při navazujících pracích je doporučováno používat Babel 7, který má však jiný ekosystém. (Nebude-li však navazující práce používat Babel vůbec, tento fakt ji nijak neovlivní, jelikož práce obsahuje i překompilovanou verzi JavaScriptových souborů.)

\subsection{Testovací prostředí}
Pro testovací prostředí byl vybrán framework AVA \cite{ava} a pro zjištění pokrytí kódu je používán Istanbul Code Coverage\cite{instambul}, respektive jeho verzi pro příkazovou řádku "nyc" \cite{nyc}



\section{Vstupně výstupní formát automatu}
Formát jazyka bude nejlépe vysvětlen jeho popisným příkladem viz \ref{example:format}, kde <cokoliv>, není součástí formátu, nýbrž popis hodnoty. Důležité je pouze zmínit, že formát je v zápisu JSON, jeho struktura je odvozena od množinového zápisu automatu, a uvedený příklad popisuje povinné hodnoty pro konečný automat a je možné ho jakkoliv rozšířit, bez porušen kompatibility, při zachování stávající struktury

%Define the listing package
\begin{lstlisting}[language=JavaScript,label={example:format}, caption={Přijímaný formát konečného automatu}]
{

    "states": [
    		{name: <unikátní název stavu (řetězec)>},
     		{name:  <unikátní název stavu (řetězec)>
     		],
    "alphabet": [<symbol abecedy (řetězec)>,...],
    "rules": [
        {
            "from": {"state": {"name": <název existujícího stavu (řetězec)>},},
            "to": {"state": {"name": <název existujícího stavu (řetězec)>}},
            "symbol": <existující symbol abecedy (řetězec)>,
        },
        {
            "from": {"state": {"name": <název existujícího stavu (řetězec)>},},
            "to": {"state": {"name": <název existujícího stavu (řetězec)>}},
            "symbol": <existující symbol abecedy (řetězec)>,
        }
    ],
    "finalStates": [{name: <název existujícího stavu (řetězec)>}],
    "initialState": {name: <název existujícího stavu (řetězec)>}
}

\end{lstlisting}

V rámci rozšíření, pro zásobníkové automaty, je tento formát obohacen o následující klíče:
\begin{lstlisting}[language=JavaScript,label={example:format}, caption={Rozšíření pro zásobníkový automat}]
{
	"rules": [
		{
			"from": {
				"state": {"name": <název existujícího stavu (řetězec)>},
				"stackTop": <symbol co musí být na vrcholu zásobníku (řetězec délky 1)>
			},
			"to": {
				"state": {"name": <symoy které budou na vrcholu zásobníku (řetězec délky 0-2)>}
			},
			"symbol": <existující symbol abecedy (řetězec)>,
		},
	],
	"initialStackSymbol": <(řetězec o délce 1)>,
	"stackAlphabet": [<symbol zásobníkové abecedy(řetězec o délce 1)>,...]
}

\end{lstlisting}

Zde je důležité, dát si velký pozor na to, že symbol zásobníkové abecedy musí být řetězec délky jedna. 

\section{Implementace}
Samotná implementace je provedena třídně objektově, kde každá součást automatu a automat sám je objekt, obdobně jako je tomu u formátu výše. Nejednoduší popsání bude pomocí třídní reprezentace viz \ref{example:hierarchy}, obdobně jako výše popsaný formát.


% Using typewriter font: \ttfamily inside \lstset


\begin{lstlisting}[language=JavaScript,label={example:hierarchy}, caption={Třídní hierarchie implementace}]
class Automata = {
    states: { <název stavu (řetězec)>: State };
    alphabet: Alphabet;
    rules: Rule[]; //pole Rule
    initialState: State;
    finalStates: { <název stavu (řetězec)>: State };
}

class State = {
    name: <(řetězec)>;
    isInitial: boolean;
    isFinal: boolean;
    isNonterminating: boolean;
}

class Alphabet = <Pole s pouze unikátními prvky (pole)>

class Rule = {
    from: {state:State};
    to: {state:State};
    symbol: string;
}

\end{lstlisting}



\section{Implementace libovolné operace nad tímto ekosystémem}
Každá operace jenž je možné provést nad daným automatem je pro tento typ automatu pomocí implementovatelná.
Tato kapitola se bude věnovat implementace hypotetické operace.

Představme si tedy unární operaci $allowEmpty$ jejímž cílem je pouze donutit automat aby přijímal prázdný řetězec.
(viz \ref{example:allowEmptyES6} a \ref{img:allowEmptyES5})


\begin{lstlisting}[language=JavaScript,label={example:allowEmptyES6}, caption={Ukázka implementace operace za použití EcmaScript 2015 a Flow}]
//@flow
import FA from "../Automata/FA/FA.js";

/**
* Generuje automat přijímající prázdný řetězec
* @param {FA} automata
*/
export default function allowEmpty(automata:FA):FA {
	let newAutomata = automata.copy();
	newAutomata.initialState.isFinal = true;
	return newAutomata;
}
\end{lstlisting}

\begin{lstlisting}[language=JavaScript,label={img:allowEmptyES5}, caption={Ukázka implementace operace pomocí standardního JavaScriptu}]
var FA = require("../Automata/FA/FA.js").default;

/**
* Generuje automat přijímající prázdný řetězec
* @param {FA} automata
*/
function allowEmpty(automata) {
	var newAutomata = automata.copy();
	newAutomata.initialState.isFinal = true;
	return newAutomata;
}
window.exports = {default:allowEmpty};
\end{lstlisting}


K použití operace pak už pouze stačí, vytvořit automat a zavolat tuto operaci tak, že automat je jejím argumentem. (příklad viz \ref{example:Use})

\begin{lstlisting}[language=JavaScript,label={example:Use}, caption={ukázka použití operace}]
import FA from "../Automata/FA/FA.js";
import allowEmpty from 'allowEmpty.js';
let automata = new FA(/*...*/);
let newAutomata = allowEmpty(automata);
----------------------------------------------------------------------
var FA = require("../Automata/FA/FA.js").default;
var allowEmpty = require('allowEmpty.js').default;
var automata = new FA(/*...*/);
var newAutomata = allowEmpty(automata);
\end{lstlisting}

\section{Zprovoznění}
Pro zprovoznění práce je nutné mít nainstalováno následující:
\begin{enumerate}
	\item \textit{node} verze 9.11.1 a vyšší \url{https://nodejs.org/en/download/}
	\item \textit{npm} verze 5.8.0 a vyšší \url{https://www.npmjs.com/get-npm}, 
\end{enumerate}
Práce jako taková může běžet na serveru EVA, bez potřeby jakýchkoliv doplňků.


\textbf{Zprovoznění pro použití:}
\begin{enumerate}
	\item Získání kódů bakalářké práce
	\item cd \{havní složka práce\}
	\item npm install --production
\end{enumerate}
\textbf{Zprovoznění pro úpravu:}
\begin{enumerate}
	\item Získání kódů bakalářké práce
	\item cd \{havní složka práce\}
	\item npm install
\end{enumerate}

\textbf{Spuštění testů:}
Zprovozníme pro úpravu, můžeme použít i zprovoznění pro použití, ale pak je potřeba: \textit{npm install ava}\\
Máme li zprovozněno pouze pro použití s AVA, můžeme spustit testy pomocí \textit{npm test}, Máme li zprovozněno pro úpravu, můžeme použít nasledující příkazy:
\begin{enumerate}
	\item \textit{npm test} \#spustí testy
	\item \textit{npm run build} \#vygeneruje novou složku \textbf{dist} dle změn v \textbf{src}
	\item \textit{npm run testbuild} \#spustí build a následně testy
	\item \textit{npm run testES6} \#spustí testy nad src
	\item \textit{npm run coverage} \#spustí testy s "code coverage"
	\item \textit{npm run lcovCoverage} \#spustí testy s "code coverage", které generují \textit{coverage/lcov-report/index.html}, což je hezky proklikatelná verze "code coverage"
\end{enumerate}