\section{Průnik (Intersection)}
\subsection{Provedení nad automatem}
Průnik dvou Automatů je jednoduchá, avšak dosti zdlouhavá operace. Její aplikaci si předvedmě na příkladu.
Mějme dva automaty :

$M=\{Q_{M}, \Sigma_{M}, R_{M},s_{M}, F_{M}\}$ a $N=\{Q_{N}, \Sigma_{N}, R_{N},s_{N}, F_{N}\} $

Průnik poté provedeme následovně:
\begin{equation}
\label{eqA:Intersection2}
\begin{split}
    Intersection(M,N) = \{ &\\
          &Q = Q_{M} \times Q_{N},\\
    & \tab   \Sigma = \Sigma_{M} \cap \Sigma_{N} \\
    & \tab   R = \{ (q_{M},q_{N})\alpha \longrightarrow (q_{M}\alpha, q_{N}\alpha);\\
    & \tab\tab q_{M} \in Q_{M} \land q_{N} \in Q_{N} \land \alpha \in \Sigma\\
    & \tab\}, \\
    & \tab s = (s_{M}, s_{N}), \\
    & \tab F = F_{M} \times F_{N} \\
    \}&
\end{split}
\end{equation}

\subsection{Implementace}
Implementace zde pouze opisuje postup vytvoření automatu. (Soubor \textit{src/operations/intersectionFA.js})
