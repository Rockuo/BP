\section{Shuffle}
Operaci shuffle můžeme definovat jako všechny kombinace které vzniknou pomícháním znaků z řetězců u a v se zachováním pořadí znaků tak jak byly v původním řetězci. V podstatě bychom tuto operaci mohli popsat jako prolnutí dvou řetězců. (viz \ref{eq:Shuffle}) Tuto operaci potom můžeme generalizovat tak na jazyky, viz \ref{eq:ShuffleLang}. Znalost a pochopení této operace nám pomůže v chápání dalších operací, jako je vkládání(Kapitoly \ref{sec:seqIns} a \ref{sec:parIns}).


\begin{equation}\label{eq:Shuffle}
\begin{split}
Shuffle(u,v) = \{& \\
& u_{1}v_{1}...u_{i}v_{j}|\\
&\tab u=u_{1}...u_{i} \land u \in \Sigma* \\
&\tab v=v_{1}...v_{j} \land v \in \Sigma*\\
&\tab u_{p} \in u \land u_{p} \in \Sigma \cup \epsilon;\\
&\tab v_{q} \in v  \land v_{q} \in \Sigma \cup \epsilon;\\
&\tab 1 \leq p \leq i \land 1 \leq q \leq j\\
\}&
\end{split}
\end{equation}

\begin{equation}\label{eq:ShuffleLang}
Shuffle(K,L) = \{w|w=Shuffle(u,v); u \in K \land v \in L \}
\end{equation}


\subsection{Vlastnosti}
Operace shuffle je uzavřená nad regulárními jazyky a neuzavřená nad jazyky bezkontextovými.

\subsection{Příklad}
\textbf{Operace Shuffle nad dvěma řetězci}:\\
Nechť existují řetězce $u$ a $v$, ta že $u = "ab", v = "cd"$ 
$Shuffle(u,v)$ se poté bude rovnat $\{"abcd", "acbd", "acdb", "cabd", "cadb", "cdab"\}$
