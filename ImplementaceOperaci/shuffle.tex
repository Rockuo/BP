\section{(Shuffle)}
\subsection{Provedení nad automatem}
Aplikace se z počátku může zdát složitá. Uvědomíme-li si však, že můžeme používat složení automatů v kartézském součinu pro sledování cesty v kterém automatu se pohybujeme, zjistíme, že stačí provést pouze následující:\\
Mějme dva automaty :

$M=\{Q_{M}, \Sigma_{M}, \delta_{M},s_{M}, F_{M}\}$ a $N=\{Q_{N}, \Sigma_{N}, \delta_{N},s_{N}, F_{N}\} $

Promíchání poté provedeme následovně:
\begin{equation}
\label{eqA:Intersection}
\begin{split}
Shuffle(M,N) = \{ &\\
 & Q = Q_{M} \times Q_{N},\\
 &   \Sigma = \Sigma_{M} \cup \Sigma_{N} \\
    &   \delta = \{ \\
    & \tab (q_{M},q_{N})\alpha \longrightarrow (q_{M}\alpha, q_{N}) \vee (q_{M},q_{N}\alpha);\\
    & \tab q_{M} \in Q_{M} \land q_{N} \in Q_{N} \land \alpha \in \Sigma\\
    & \tab \}, \\
    &  s = (s_{M}, s_{N}), \\
    & F = F_{M} \times F_{N} \\
    & \}
\end{split}
\end{equation}

\subsection{Implementace}
Jak je výše uvedeno v zápisu automatu, promíchání probíhá tak, že se posouváme v dle pravidla automatu $M$, nebo automatu $N$. Viz následující ukázka kódu, kde si ukážeme generování pravidel pro Shuffle.
\begin{lstlisting}[language=JavaScript, caption={Ukázka generování přechodů pro shuffle (\textit{src/operations/shuffleFA.js})}, label=example:shuffle]
/**Anotace*/
function createRules(
	left: FA,
	right: FA,
	newStates: { [key: string]: MergedState }
): Rule[] {
	let newRules = [];
	//pro každý nový stav generujeme pravidla
	for (let newState: MergedState of objectValues(newStates)) {
		//generujeme pravidla z levého automatu
		let leftRules = /*Filtrovaná pravidla levého automatu, pro tento stav*/;
		for (let rule: Rule of leftRules) {
			newRules.push(new Rule({
				from: {state: newState},
				symbol: rule.symbol,
				to: {state: newStates[MergedState.createName(rule.to.state, 	newState.oldRight)]}
			}));
		}

		//generujeme pravidla z pravého automatu
		/*Obdobně jako pro lévé*/
			...
				to: {state: newStates[MergedState.createName(newState.oldLeft, rule.to.state)]}
			...
	}
	return newRules;
}

\end{lstlisting}