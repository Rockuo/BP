\section{Operace "rozdílné" (Operation Diferent)}
\label{section:OD}
Operation Diferent, nebo-li rozdílné (neplést s Diference, rozdíl), tato operace přijímá libovolné dva jazyky a provádí nad nimi operaci kterou nejlépe definuje \ref{eq:Different}, pokud bychom ji chtěli definovat do podrobna, byla by popsána rovnicí \ref{eq:Different_detail}

\begin{equation}\label{eq:Different}
Different(L_{1}, L_{2}) = Union(L_{1}, L_{2}) - Intersection(L_{1}, L_{2})
\end{equation}


\begin{equation}\label{eq:Different_detail}
Different(L_{1}, L_{2}) = \{w| (w\in L_{1}\land  w\notin L_{2}) \lor (w\notin L_{1}\land  w\in L_{2})\}
\end{equation}

\subsection{Vlastnosti}
Vzhledem k faktu, že tato operace jde rozložit na několik jednodušších operací, jsme tak schopni odvodit i chování této operace. Vzhledem k tomu, že všechny operace uvedené v \ref{eq:Different} jsou uzavřené nad regulárními jazyky, můžeme si říci, že celá operace je nad regulárními jazyky uzavřená. Stejným způsobem víme, že tato operace není uzavřená nad bezkontextovými jazyky.

Operace při opakovaném volání nevykazuje žádné speciální vlastnosti, vyjma toho, že jakékoliv použití této operace nad rodinou jazyků nám zaručí existenci prázdného jazyku. Existence prázdného jazyku nám dále zaručí, že v dalším výsledku nám nezmizí žádný jazyk a tedy je výsledná rodina vždy stejná, nebo větší.Po určitém počtu iterací se nakonec dostáváme k rodině, která je nad touto operací tranzitivně uzavřená
