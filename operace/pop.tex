\section{Pop}
Operace Pop, de facto operace reverzní ke konkatenaci. Tuto operaci si rozdělíme na $LPop(K,L)$ a $RPop(K,L)$. Toto rozdělení provádíme z toho důvodu, že konkatenovat můžeme z obou stran, a tedy bychom rádi z obou stran i odebírali, což musíme nějak specifikovat. Obě operace pop jsou každá, vlastním způsobem omezená operace Sekvenční mazání.
Obě operace si můžeme definovat následujícím způsobem:

\begin{equation}\label{eq:lPop}
	LPop(K,L) = \{y|xy |in K \land x \in L\}
\end{equation}


\begin{equation}\label{eq:rPop}
	RPop(K,L) = \{x|xy |in K \land x \in L\}
\end{equation}

\subsection{Vlastnosti}
\begin{theorem}[Uzavřenost lPop nad regulárními jazyky] \label{thm:lPopReg}
	Libovolné dva regulární jazyky, jsou uzavřené nad operací lPop.
\end{theorem}

\begin{proof}[Důkaz \ref{thm:lPopReg}]
	Nechť existují dva jazyky $L_{1}$ a $L_{2}$, nad abecedou $\Sigma$ a nechť existuje automat $M=\{Q,\Sigma,R,s,F\}$ přijímající jazyk $L_{1}$. Pro každé dva stavy $s,q \in Q$  nechť existuje:
	$$L_{s,q} = \{w|sw\longrightarrow^{*}q \in M; w\in \Sigma\}$$
	 Poté uvažujme automat:
	 $$M^{'}=\{Q,\Sigma\cup\{\#\},R^{'},s,F\}$$
	 kde $$R^{'} = R\cup\{s\#\longleftarrow q|s,q\in Q \land L_{2} \cap L_{s,q} \neq \emptyset\}$$
	 kde $\#$ je nový symbol nepatřící do abecedy $\Sigma$.
	 
	 Následně můžeme tvrdit že:
	 $$LPop(L_{1},L_{2}) = h(L(M^{'} \cap \Sigma^{*}\#\Sigma^{*}))$$
	 
	 Následně důkaz pokračuje, jako důkaz Sekvenčního Mazání, viz \cite{lilaKari} strany 60-61, kde $A^{'}=M^{'}$.
\end{proof}

\begin{theorem}[Uzavřenost rPop nad regulárními jazyky] \label{thm:rPopReg}
	Libovolné dva regulární jazyky, jsou uzavřené nad operací rPop.
\end{theorem}

\begin{proof}[Důkaz \ref{thm:lPopReg}]
	Nechť existují dva jazyky $L_{1}$ a $L_{2}$, nad abecedou $\Sigma$ a nechť existuje automat $M=\{Q,\Sigma,R,s,F\}$ přijímající jazyk $L_{1}$. Pro každé dva stavy $q \in Q \land f \in F$  nechť existuje:
	$$L_{q,f} = \{w|qw\longrightarrow^{*}f \in M; w\in \Sigma\}$$
	Poté uvažujme automat:
	$$M^{'}=\{Q,\Sigma\cup\{\#\},R^{'},s,F\}$$
	kde $$R^{'} = R\cup\{s\#\longleftarrow q|s,q\in Q \land L_{2} \cap L_{q,f}\}$$
	kde $\#$ je nový symbol nepatřící do abecedy $\Sigma$.
	
	Následně můžeme tvrdit že:
	$$LPop(L_{1},L_{2}) = h(L(M^{'} \cap \Sigma^{*}\#\Sigma^{*}))$$
	
	Následně důkaz pokračuje, jako důkaz Sekvenčního Mazání, viz \cite{lilaKari} strany 60-61, kde $A^{'}=M^{'}$.
\end{proof}

Při používání této operace, je důležité mít na paměti, že i-když o ní můžeme uvažovat jako o opaku konkatenace, tedy $RPop(Concatenation(K,L),L) = K$, tak $Concatenation(RPop(K,L),L) \neq K$. A to protože, do $Pop$ operací nepřispívají všechny řetězce jazyka K, nýbrž pouze ty, ze kterých můžeme něco smazat.

\subsection{Příklad}
\textbf{Ukázka použití LPop a RPop:}\\
Nechť existují dva jazyky, $K=\{aba,ab,c\}$ a $L=\{a\}$.

Použití operace $LPop$ nám generuje jazyk $K^{'}=LPop(K,L) =\{ba,b\}$ a použití operace $RPop$ nám generuje jazyk $K^{''}=RPop(K,L) =\{ab\}$

\textbf{Ukázka $\mathbf{RPop(Concatenation(K,L),L) = K}$:}\\
Uvažujme stejné jazyky jako v předchozím příklade, konkatenace $Concatenation(K,L)$ nám generuje jazyk $K^{'}=\{abaa,aba,ca\}$. Následné použití RPop nám generuje opět jazyk $K$, $RPop(K^{'}, L) = K$.

\textbf{Ukázka $\mathbf{Concatenation(RPop(K,L),L) \neq K}$:}\\
Opět uvažujme stejné jazyky jako v předchozích příkladech, použití operace $RPop(K,L)$ nám generuje jazyk $K^{'}={ab}$, což se zcela zřetelně nerovná $K$