\section{Doplněk (Complement)}
Doplněk jazyka si můžeme definovat následujícím příkladem:
Představme si, že máme jazyk $L_{1}$ patřící do abecedy $\Sigma$. Doplněk jazyka $L_{1}$, jsou všechny řetězce patřící do množiny řetězců $\Sigma^{*}$ a zároveň nepatřící do jazyka $L_{2}$

Čistě pro představu si můžeme říci, že $Complement(L_{1})=\Sigma^{*} - L_{1}$.

\subsection{Vlastnosti}
Doplněk jazyka je operace uzavřená nad jazyky regulárními a neuzavřená nad jazyky bezkontextovými.

Také je zcela zřejmé, že $Complement(Complement(L_{1})) = L_{1}$.