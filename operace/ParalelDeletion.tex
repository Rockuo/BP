\section{Paralel deletion}
Paralelní mazání je jedna z mnoha operací soustředících se na mazání jednoho řetězce z řetězce druhého. Narozdíl od Sequential deletion (viz TODO reference), která maže výskyt řetězce u v řetězci v avšak s tím rozdílem, že maže všechny nepřekrývající se výskyty řetězce u. (viz [todo dodělat roovnici, je definována v http://www.fit.vutbr.cz/research/pubs/TR/2005/sem\_uifs/s050121clanek.pdf, ale tomu nerozumim => pochopit => zjednodušit na 2 řetězce, bez použití jazyku => zobecnit na použití nad rodinou jazyků]) 

\subsection{Vlastnosti}
TODO: vlastnosti při iterovaném použití
Vlastnosti nad regulárními, bezkontextovými jazyky
vlastnosti nad jazyky obsahující řetězce nakonečné délky

\section{Operation SubAlpha}
Toto je jen pomocná operace, kterou si definujeme písemně pro další použití. Operace SubAlpha vrací všechny znaky které řetězec, nebo jazyk obsahuje. (Viz příklad)

\subsection{Příklad}
$$\Sigma_{1} = \{"a", "b", ..., "z"\}, L = \{"abc", "ade"\}$$

z čehož plyne 

$$u,v \in \Sigma_{1}$$. 

Provedeme-li nad L operaci SubAlpha získáme: 

$$\Sigma_{2} = SubAlpha(L) = \{"a", "b", "c", "d", "e"\}$$

a můžeme též tvrdit 

$$Sigma_{2} \subseteq Sigma_{1}$$

