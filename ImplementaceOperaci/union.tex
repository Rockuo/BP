\section{Sjednocení (Union)}
Návod jak provést tuto operaci je předveden v předmětu IFJ [TODO reference] a proto se podíváme pouze na její implementaci.
\subsection{Implementace}
V této operaci si ukážeme, nejen jak provést sjednocení, ale jak ho provést nad několika typy automatů. Proto, ikdyž se jedná o relativně jednoduchou operaci, ukážeme si zde její kód.

\begin{lstlisting}[language=JavaScript,label={example:union}, caption={Ukázka implementace operace Union (\textit{src/operations/union.js})}]
{
//@flow
/*Import ptřebných věcí*/

/**
* Sjednocení dvou Automatů
* @param {Automata} left
* @param  {Automata} right
* @return Automata
*/
export default function union(left: (Automata | PA | FA), right: (Automata | PA | FA)) {
	//Převedeme si automat na jeho serializovatelnou reprezentaci
	let plainLeft: T_AnyPlainAutomata = toPlainLeft(left),
			plainRight: T_AnyPlainAutomata = toPlainRight(right);

	// suffix pro nové stavy
	let suffix = State.randomName();


	//sestrojíme automat
	let plainUnion:T_AnyPlainAutomata = {
			states: [{name: 's' + suffix}, /*původní stavy obou automatů*/],
			alphabet: [./*sjednocení abeced*/],
			finalStates: [/*sjednocení koncových stavů*/],
			initialState: {name: 's' + suffix}
	};

	// wrapper pro předávání parametrů
	let paFun = /*funkce pro zásobníkový automat*/;
	// wrapper pro předávání parametrů
	let faFun = /*funkce pro konečný automat*/;

	// Nastavíme jaká funkce se má volat pro kterou kombinaci automatů
	//v tomto případě, pokud je jeden z automatů Zásobníkový, používáme funkci additionalPA
	return overload(
		[
			{parameters: [{value: left, type: FA}, {value: right, type: PA}], func: paFun},
			{parameters: [{value: left, type: PA}, {value: right, type: FA}], func: paFun},
			{parameters: [{value: left, type: PA}, {value: right, type: PA}], func: paFun},
			{parameters: [{value: left, type: FA}, {value: right, type: FA}], func: faFun},
		]
	);
}

\end{lstlisting}