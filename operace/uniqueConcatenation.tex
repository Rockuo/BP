\section{Unikátní konkatenace (Unique Concatenation)}
Unique Concatenation, nebo-li Výlučná konkatenace, je operace jenž se chová jako konkatenace dvou jazyků, až na ten fakt, že nepřijímá takové řetězce z K a L, jejichž konkatenace by patřila do K nebo L.
Pro zjednodušení si určeme konkatenaci, tak, že konkatenace je operace popsatelná rovnicí \ref{eq:Concatenation}.
Unikátní konkatenace poté je definována rovnicí \ref{eq:UniqueConcatenation_detail}.
Ve zkratce výsledek unikátní konkatenace tytéž řetězce jako konkatenace normální, s výjimkou, že neobsahuje řetězce které již byly obsaženy v jazicích jenž jsme konkatenovali, což je nejlépe popsáno rovnicí  \ref{eq:UniqueConcatenation}.

\begin{equation}\label{eq:Concatenation}
Concatenation(L_{1}, L_{2}) = \{w|w=xy; y\in L_{1}\land  y\notin L_{2}\}
\end{equation}

\begin{equation}\label{eq:UniqueConcatenation_detail}
UniqueConc(L_{1}, L_{2}) = \{w|w=xy; y\in L_{1}\land  y\notin L_{2} \land xy \notin L_{1} \land xy \notin L_{2} \}
\end{equation}

\begin{equation}\label{eq:UniqueConcatenation}
UniqueConc(L_{1}, L_{2}) = Concatenation(L_{1}, L_{2}) - L_{1} - L_{2}
\end{equation}

\subsection{Vlastnosti}
Díky smutné neunikátnosti této operace jsme schopni z jejího vyjádření \ref{eq:UniqueConcatenation} vyvodit, že bude uzavřená nad regulárními a neuzavřená nad bez kontextovými jazyky.

Rodina vzniklá touto operací je velikosti $n^2$ nebo $(n-1)^2+1$ pokud původní rodina obsahovala prázdný jazyk. 
