\chapter{TODO Lepší název Další vlastností operací}
V předchozí kapitole jsme si ukazovali několik zajímavých operací a jejich vlasti. Ne však všechny vlastnosti jsou tak zřejmé, snadno dohledatelné, nebo vůbec dokázané. V této kapitole se podíváme na tyto vlastnosti, případně pouze domělé nedokázané vlastnosti.(TODO ta věta nějak nesedí)

\section{Vlastnosti operace Diferent Union}
[viz sekce \ref{se:DiferentUnion}]
Už jsme si probrali nějaké základní zajímavé vlastnosti této operace, které se vyskytovali při aplikaci na rodinu jazyků. Nyní se podívejme co může nastat, přimíchá-li se nám sem nekonečno. (Poznámka: Veškerá uzavřenost této operace je totožná s uzavřeností nad operací Union)

\subsection{Použití operace nad dvěma jazyky}
Při použití operace nad dvěma jazyky, kdy alespoň jeden z nich je nekonečný, nevznikají žádné zvláštní rozdíly oproti použití nad konečnými jazyky. Výsledkem je vždy jeden jazyk, který je sjednocením použitých jazyků.

\subsection{Konečná rodina jazyků}
Pro konečnou rodinu konečných jazyků jsme si již vlastnosti ukázali, ale co se stane, je-li jeden či více jazyků v rodině nekonečných?

Překvapivě, operace se bude chovat relativně standardně (tady obdobně jako u konečných jazyků). Zajímavostí de je to, že konečné vstupní $SL$ se může stát nekonečným $SL$, které může být menší než výstupní nekonečné $BL$

\subsection{Nekonečně velká rodina jazyků}
Na tento případ se opět vztahují výše zmíněné vlastnosti s jednou výjimkou, libovolně velká iterace nemůže z nekonečně velké rodiny udělat rodinu konečnou a tedy se nikdy nedostaneme do stavu, kdy $FS = 0$.

Zde si také můžeme ukázat, jednu zajímavou vlastnost rodin jazyků. Běžné rodiny jazyků nejsou nekonečné. Pokud bychom chtěli vzít nekonečně velkou rodinu jazyků, tak logicky musí obsahovat nekonečně mnoho jazyků nekonečné délky, respektive se dostáváme do stavu, kdy cokoliv jiného je velice složitě představitelné. Avšak zde by se pozorovatel zmýlil. Mi jsme schopni vytvořit nekonečně velkou rodinu jazyků skládajících pouze z konečně velkých jazyků. Za pomoci nekonečně velké abecedy jsme totiž schopni vytvořit nekonečně velkou rodinu jazyků která bude mí $SS = BS = 1$.
