\section{Rozdíl (Difference)} \label{sec:diff}
Rozdíl dvou jazyků je dosti podobný rozdílu dvou množin, tedy pokud $L_{3}=Difference(L_{1},L_{2})$, tak $L_{3}$ obsahuje všechny řetězce patřící do jazyka $L_{1}$ a zároveň nepatřící do jazyka $L_{2}$, což si můžeme znázornit rovnicí \ref{eq:differenceAbs}. Mnohem výstižněji to však můžeme popsat rovnicí \ref{eq:differencePerf}


\begin{equation}
\label{eq:differenceAbs}
	Difference(L_{1}, L_{2})= {w|w\in L_{1} \land w \notin L_{2}}
\end{equation}

\begin{equation}
\label{eq:differencePerf}
Difference(L_{1}, L_{2})= Intersection(L_{1},Complement(L_{2}))
\end{equation}

\subsection{Vlastnosti}
Z rovnice \ref{eq:differencePerf} nám jasně plyne, že rozdíl je uzavřen nad regulárními a neuzavřen nad bezkontextovými jazyky, neb průnik i doplněk jsou oba uzavřené nad regulárními a neuzavřené nad bezkontextovými.