\section{Rozdílné sjednocení (Different Union)}\label{se:DiferentUnion}
Tato operace vychází z operace sjednocení, avšak s tím rozdílem, že povoluje pouze sjednocení nestejných jazyků. (Viz. \ref{eq:DifferentUnion})

\begin{equation}\label{eq:DifferentUnion}
DifferentUnion(L_{1}, L_{2}) = \{w|(w\in L_{1}\lor w\in L_{2}) \land L_{1} \neq L_{2}\}
\end{equation}

\subsection{Vlastnosti}
Operace je uzavřena nad regulárními, ne však nad bezkontextovými jazyky.

\begin{theorem}[Uzavřenost nad regulárními jazyky] \label{thm:DUReg}
Použití operace DifferentUnion nad libovolnými dvěma regulárními jazyky $K$ a $L$ generuje opět regulární jazyk.
\end{theorem}

\begin{proof}[Důkaz \ref{thm:DUReg}]
Nejprve musíme zjistit, jak porovnat dva jazyky $K$ a $L$, což provedeme následovně

$K \neq L \Longleftrightarrow K-L\neq\varnothing \land L-K\neq\varnothing$



Pravou stranu si dosadíme do rovnice \ref{eq:DifferentUnion} a jelikož víme, že operace Sjednocení i Rozdíl jsou uzavřeny nad rekulárními jazyky, tedy operace DifferentUnion je uzavřena nad regulárnímy jazyky.
\end{proof}

\begin{theorem}[Uzavřenost nad Bezkontextovými jazyky jazyky] \label{thm:DUNon}
Předpokládejme že použití operace DifferentUnion nad libovolnými dvěma bezkontextovými jazyky $K$ a $L$ generuje opět bezkontextový jazyk.
\end{theorem}

\begin{proof}[Důkaz \ref{thm:DUNon}]
Postupujeme stejně jako u důkazu \ref{thm:DUReg}, avšak zjišťujeme, že operace Rozdíl není uzavřena nad bezkontextovými jazyky. Tedy ani operace DifferentUnion nemůže být nad těmito jazyky uzavřena, neb nemůžeme tvrdit, že jsme schopni porovnat libovolné dva bezkontextové jazyky a rozhodnout o jejich rovnosti.
\end{proof}

Při použití nad rodinou jazyků nám vzniká opačný efekt než kupříkladu u sjednocení a to, že může vzniknout rodina menší než rodina nad kterou jsme operaci aplikovali. Velikost rodiny zde opět záleží na tom, zda-li rodina obsahuje jazyk jenž je podmnožinou jiného jazyka. Při opakovaném použití se však vždy dostaneme do stavu kdy se rodina začne zmenšovat do bodu kdy obsahuje jeden jazyk a nakonec tedy je tato rodina prázdná. Tato vlastnost platí i pro nekonečně velké rodiny jazyků, u nichž však k nule nikdy nedojdeme a rodina se nám zmenšuje a zároveň zůstává nekonečná. (Různě velká nekonečna obdobně jako množina celých čísel je menší nekonečno než množina čísel reálných.)

Tento efekt nastává proto, že pokud se nemůže jazyk sjednotit sám se sebou a ani se svou podmnožinou, nemá možnost být ve výsledku další rodiny a tedy je "vyloučen".

Další vlastností které je dobré si všimnout, je že jazyky v rodině jazyků se vždy zvětšují a to proto, že se do výsledku dostanou pouze jazyky jenž jsou sjednocením dvou jiných, nebo jazyky jež jsou nadmnožinou jazyka jiného.

\subsection{Příklad}
Na příkladu si můžeme všimnout, kolísání velikosti rodin jazyků. Po prvním použití je vidět, že se velikost výsledné rodiny oproti předchozí zvětší a to díky jazyku $L_{5}$ jenž je podmnožinou všech jazyků. Následně se velikost rodin začíná zmenšovat až nakonec dojdeme do stavu, kdy je rodina jazyků prázdná.\\


\textbf{Iterativně použitá operace DifferentUnion nad rodinou jazyků}\\
Mějme rodinu jazyků $R_{1}$ a postupně aplikujme operaci DifferentUnion, dokud se nedostaneme k prázdné rodině jazyků

\begin{equation*}
	\begin{split}
	R_{1} = \{&L_{1}:\{0,00,000\},L_{2}:\{1,11,111\},L_{3}:\{0,1\},L_{4}:\{\epsilon\},L_{5}:\{\}\}\\
		R_{2} = \{&\\
		&L_{1}:\{0,00,000\},L_{2}:\{1,11,111\},L_{3}:\{0,1\},L_{4}:\{\epsilon\},\\
		&L_{6}:\{0,00,000,1,11,111\}, L_{7}:\{0,00,000,1\},\\
		&L_{8}:\{0,00,000,\epsilon\}, L_{9}:\{0, 1, 11,111\}, L_{10}:\{1, 11,111,\epsilon\},\\
		&L_{11}:\{0,1,\epsilon\}\\
		\}&\\
		R_{3} = \{&\\
		&L_{6}:\{0,00,000,1,11,111\}, L_{7}:\{0,00,000,1\},\\
		&L_{8}:\{0,00,000,\epsilon\}, L_{9}:\{0, 1, 11,111\}, L_{10}:\{1, 11,111,\epsilon\},\\ &L_{11}:\{0,1,\epsilon\},L_{12}:\{0,00,000,1,11,111,\epsilon\},\\
		&L_{13}:\{0,00,000,1,\epsilon\},L_{14}:\{0,1,11,111,\epsilon\}\\
		&\}\\
		R_{4} =\{&\\
		&L_{6}:\{0,00,000,1,11,111\},L_{12}:\{0,00,000,1,11,111,\epsilon\},\\
	&L_{13}:\{0,00,000,1,\epsilon\},L_{14}:\{0,1,11,111,\epsilon\}\\
	&\}\\
	R_{5} = \{&L_{12}:\{0,00,000,1,11,111,\epsilon\}\}	\\
	R_{6} = \{&\}
	\end{split}
\end{equation*}