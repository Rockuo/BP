% !TeX spellcheck = cs_CZ
%=========================================================================
% (c) Michal Bidlo, Bohuslav Křena, 2008

\newcommand\tab[1][1cm]{\hspace*{#1}}

\chapter{Úvod}
TODO definice jak se operuje nad množinami jazyků
Několik vlastností, které budeme pozorovat pokud bude možno. Délka Nejdelšího řetězce v jazyku(dále jen $BS_{L}$), nebo v rodině(dále jen $BS_{F}$), délka nejkratšího řetězce v jazyku(dále jen $SS_{L}$), případně v rodině(dále jen $SS_{F}$).
Velikost nejdelšího jazyka v rodině(dále jen $BL$), velikost nejkratšího jazyka v rodině(dále jen $SL$) a nakonec velikost samotné rodiny(dále jen $FS$).

\chapter{Operace} \label{chap:Operace}
V této kapitole si představíme existující a definujeme si nové operace nad různými množinami(rodinami) jazyků.
Operace si ukážeme, nad nimi zobrazíme příklady a řekneme si jejich známé vlastnosti. Na příkladech existujících a známých operací budeme poté stavět při představování operací nových. Na vlastnosti složitějších operací, jenž bude potřeba dokazovat se podíváme v další kapitole.

\section{Sjednocení (Union)}
Union neboli sjednocení je známá operace kdy sjednocujeme dva jazyky A a B a vzniká nám jazyk C obsahující prvky jazyka A i B. (viz. \ref{eq:Union})

\begin{equation}\label{eq:Union}
Union(L_{1}, L_{2}) = \{w|w\in L_{1}\lor w\in L_{2}\}
\end{equation}

\subsection{Vlastnosti}
Operace uzavřena nad Regulárními a bezkontextovými jazyky. Tedy pokud operaci použijeme nad dvěma regulárními jazyky, výsledkem bude regulární jazyk a obdobně je tomu tak s bezkontextovými. Použijeme-li operaci nad rodinou regulárních jazyků, vznikne nám jiná, větší rodina regulárních jazyků.

Není-li žádný z jazyků v rodině podmnožinou jazyka v téže rodině, tak velikost vzniklé rodiny je dána rovnicí \ref{eq:triangleNumber} (čteme n plus jedna, nad dvěma), kde $n$ je délka původní rodiny. Pokud rodina obsahuje jazyky jenž jsou podmnožinou jazyků v téže rodině, je délka vzniklého jazyka na tomto faktu závislá a proto obecně víme pouze to, že bude menší než hodnota daná rovnicí \ref{eq:triangleNumber}. Je-li však mezi jazyky v rodině taková vazba, že provedení operace $Union$ nad libovolnými dvěma jazyky rodiny nám vytvoří jazyk patřící do této rodiny, víme že  vzniklá rodina jazyků bude totožná s rodinou nad kterou jsme tuto operaci používali. (Viz. podkapitola \ref{sec:UnionExample}) Tento vztah budeme nazývat, že je rodina tranzitivně uzavřena nad operací.

\begin{equation}\label{eq:triangleNumber}
\binom{n+1}{2}
\end{equation}

\subsection{Příklad}\label{sec:UnionExample}
\textbf{Opakované použití operace Union s prázdným jazykem:}\\
Mějme rodinu jazyků $R_{1}$
$$
	R_{1} = \{L_{1}:\{0,00,000\}, L_{2}:\{1ř,11,111\}, L_{3}:\{0,1\}, L_{4}:\{\epsilon\}, L_{5}:\{\}\}
$$
Použijme nad touto rodinou sjednocení: $R_{2} = Union(R_{1})$

\begin{align*}
R_{2} = \{&L_{1}:\{0,00,000\}, L_{2}:\{1,11,111\}, L_{3}:\{0,1\}, L_{4}:\{\epsilon\}, \\
&L_{5}:\{\}, L_{6}:\{0,00,000,1,11,111\}, L_{7}:\{0,00,000,1\},\\
&L_{8}:\{0,00,000,\epsilon\}, L_{9}:\{0, 1, 11,111\}, L_{10}:\{1, 11,111,\epsilon\},\\ &L_{11}:\{0,1,\epsilon\}
&\}
\end{align*}

A použijme-li sjednocení ještě jednou, získáme tranzitivně uzavřenou rodinu:\\
\tab$R_{3} = Union(R_{2})$

\begin{align*}
R_{3} = \{&L_{1}:\{0,00,000\}, L_{2}:\{1,11,111\}, L_{3}:\{0,1\}, L_{4}:\{\epsilon\}, \\
&L_{5}:\{\}, L_{6}:\{0,00,000,1,11,111\}, L_{7}:\{0,00,000,1\},\\
&L_{8}:\{0,00,000,\epsilon\}, L_{9}:\{0, 1, 11,111\}, L_{10}:\{1, 11,111,\epsilon\},\\ &L_{11}:\{0,1,\epsilon\},L_{12}:\{0,00,000,1,11,111,\epsilon\},\\
&L_{13}:\{0,00,000,1,\epsilon\},L_{14}:\{0,1,11,111,\epsilon\}\\
&\}
\end{align*}
 
 Obdobný výsledek by nastal u jakékoliv konečné rodiny jazyků. Přesněji řečeno, obsahuje li rodina jazyků $k$ jazyků, tak nejpozději při $k-1$ iteraci se dostaneme k tranzitivně uzavřené rodině.
\section{Průnik (Intersection)}
Intersection neboli průnik je opět další známá operace na které si ukážeme jaké má vlastnosti nad rodinami jazyků.
Průnik jazyků $L_{1}$ a $L_{2}$ nám dává jazyk $L_{3}$, jenž obsahuje pouze přítomné v jazyce $L_{1}$, a zároveň jazyce $L_{2}$. (Viz. \ref{eq:Intersection} )

\begin{equation}\label{eq:Intersection}
Intersection(L_{1}, L_{2}) = \{w|w\in L_{1}\land w\in L_{2}\}
\end{equation}

\subsection{Vlastnosti}
Operace uzavřena nad regulárnímy jazyky avšak nikoliv nad bezkontextovými.
Tedy oproti Union, pokud použijeme operaci Intersection nad dvěma jazyky jež jsou bezkontextové, nemůžeme si být jisti, že výsledkem bude jazyk bezkontextový.

Při použití nad rodinou jazyků, bude opět vzniklá rodina větší, za předpokladu, že původní rodina nebyla uzavřená.

\subsection{Příklad}
Na příkladu si ukážeme opakované použití operace Intersection až do vodu kdy se dostaneme do stavu kdy platí, že je operace nad rodinou tranzitivně uzavřená, což bychom si mohly dokázat tak, že operaci použijeme znovu.

\textbf{Opakované použití operace Intersection}\\
Mějme rodinu jazyků $R_{1}$

\begin{equation*}
	\begin{split}
	R_{1} = \{&\\
		&L_{1}: \{0,1,00\}, L_{2}:\{1,11,00\}, L_{3}:\{0,00,11\},\\
	\}&
	\end{split}
\end{equation*}

Použitím operace Intersection získáme $R_{2}$: $R_{2}=Intersection(R_{1})$

\begin{equation*}
\begin{split}
	R_{2} = \{&\\
		&L_{1}: \{0,1,00\}, L_{2}:\{1,11,00\}, L_{3}:\{0,00,11\},\\
		&L_{2}: \{1,00\}, L_{2}:\{0,00\}, L_{3}:\{00,11\},\\
	\}&
\end{split}
\end{equation*}

A dalším použitím Intersection se dostáváme k tranzitivně uzavřené rodině $R_{3}$

\begin{equation*}
\begin{split}
	R_{3} = \{&\\
		&L_{1}: \{0,1,00\}, L_{2}:\{1,11,00\}, L_{3}:\{0,00,11\},\\
		&L_{4}: \{1,00\}, L_{5}:\{0,00\}, L_{6}:\{00,11\},L_{7}: \{00\}\\
	\}&
\end{split}
\end{equation*}

\section{Doplněk (Complement)}
Doplněk jazyka si můžeme definovat následujícím příkladem:
Představme si, že máme jazyk $L_{1}$ patřící do abecedy $\Sigma$. Doplněk jazyka $L_{1}$, jsou všechny řetězce patřící do množiny řetězců $\Sigma^{*}$ a zároveň nepatřící do jazyka $L_{2}$

Čistě pro představu si můžeme říci, že $Complement(L_{1})=\Sigma^{*} - L_{1}$.

\subsection{Vlastnosti}
Doplněk jazyka je operace uzavřená nad jazyky regulárními a neuzavřená nad jazyky bezkontextovými.

Také je zcela zřejmé, že $Complement(Complement(L_{1})) = L_{1}$.
\section{Rozdíl (Difference)} \label{sec:diff}
Rozdíl dvou jazyků je dosti podobný rozdílu dvou množin, tedy pokud $L_{3}=Difference(L_{1},L_{2})$, tak $L_{3}$ obsahuje všechny řetězce patřící do jazyka $L_{1}$ a zároveň nepatřící do jazyka $L_{2}$, což si můžeme znázornit rovnicí \ref{eq:differenceAbs}. Mnohem výstižněji to však můžeme popsat rovnicí \ref{eq:differencePerf}


\begin{equation}
\label{eq:differenceAbs}
	Difference(L_{1}, L_{2})= {w|w\in L_{1} \land w \notin L_{2}}
\end{equation}

\begin{equation}
\label{eq:differencePerf}
Difference(L_{1}, L_{2})= Intersection(L_{1},Complement(L_{2}))
\end{equation}

\subsection{Vlastnosti}
Z rovnice \ref{eq:differencePerf} nám jasně plyne, že rozdíl je uzavřen nad regulárními a neuzavřen nad bezkontextovými jazyky, neb průnik i doplněk jsou oba uzavřené nad regulárními a neuzavřené nad bezkontextovými.
\section{Rozdílné sjednocení (Different Union)}\label{se:DiferentUnion}
Tato operace vychází z operace sjednocení, avšak s tím rozdílem, že povoluje pouze sjednocení nestejných jazyků. (Viz. \ref{eq:DifferentUnion})

\begin{equation}\label{eq:DifferentUnion}
DifferentUnion(L_{1}, L_{2}) = \{w|(w\in L_{1}\lor w\in L_{2}) \land L_{1} \neq L_{2}\}
\end{equation}

\subsection{Vlastnosti}
Operace je uzavřena nad regulárními, ne však nad bezkontextovými jazyky.

\begin{theorem}[Uzavřenost nad regulárními jazyky] \label{thm:DUReg}
Použití operace DifferentUnion nad libovolnými dvěma regulárními jazyky $K$ a $L$ generuje opět regulární jazyk.
\end{theorem}

\begin{proof}[Důkaz \ref{thm:DUReg}]
Nejprve musíme zjistit, jak porovnat dva jazyky $K$ a $L$, což provedeme následovně

$K \neq L \Longleftrightarrow K-L\neq\varnothing \land L-K\neq\varnothing$



Pravou stranu si dosadíme do rovnice \ref{eq:DifferentUnion} a jelikož víme, že operace Sjednocení i Rozdíl jsou uzavřeny nad rekulárními jazyky, tedy operace DifferentUnion je uzavřena nad regulárnímy jazyky.
\end{proof}

\begin{theorem}[Uzavřenost nad Bezkontextovými jazyky jazyky] \label{thm:DUNon}
Předpokládejme že použití operace DifferentUnion nad libovolnými dvěma bezkontextovými jazyky $K$ a $L$ generuje opět bezkontextový jazyk.
\end{theorem}

\begin{proof}[Důkaz \ref{thm:DUNon}]
Postupujeme stejně jako u důkazu \ref{thm:DUReg}, avšak zjišťujeme, že operace Rozdíl není uzavřena nad bezkontextovými jazyky. Tedy ani operace DifferentUnion nemůže být nad těmito jazyky uzavřena, neb nemůžeme tvrdit, že jsme schopni porovnat libovolné dva bezkontextové jazyky a rozhodnout o jejich rovnosti.
\end{proof}

Při použití nad rodinou jazyků nám vzniká opačný efekt než kupříkladu u sjednocení a to, že může vzniknout rodina menší než rodina nad kterou jsme operaci aplikovali. Velikost rodiny zde opět záleží na tom, zda-li rodina obsahuje jazyk jenž je podmnožinou jiného jazyka. Při opakovaném použití se však vždy dostaneme do stavu kdy se rodina začne zmenšovat do bodu kdy obsahuje jeden jazyk a nakonec tedy je tato rodina prázdná. Tato vlastnost platí i pro nekonečně velké rodiny jazyků, u nichž však k nule nikdy nedojdeme a rodina se nám zmenšuje a zároveň zůstává nekonečná. (Různě velká nekonečna obdobně jako množina celých čísel je menší nekonečno než množina čísel reálných.)

Tento efekt nastává proto, že pokud se nemůže jazyk sjednotit sám se sebou a ani se svou podmnožinou, nemá možnost být ve výsledku další rodiny a tedy je "vyloučen".

Další vlastností které je dobré si všimnout, je že jazyky v rodině jazyků se vždy zvětšují a to proto, že se do výsledku dostanou pouze jazyky jenž jsou sjednocením dvou jiných, nebo jazyky jež jsou nadmnožinou jazyka jiného.

\subsection{Příklad}
Na příkladu si můžeme všimnout, kolísání velikosti rodin jazyků. Po prvním použití je vidět, že se velikost výsledné rodiny oproti předchozí zvětší a to díky jazyku $L_{5}$ jenž je podmnožinou všech jazyků. Následně se velikost rodin začíná zmenšovat až nakonec dojdeme do stavu, kdy je rodina jazyků prázdná.\\


\textbf{Iterativně použitá operace DifferentUnion nad rodinou jazyků}\\
Mějme rodinu jazyků $R_{1}$ a postupně aplikujme operaci DifferentUnion, dokud se nedostaneme k prázdné rodině jazyků

\begin{equation*}
	\begin{split}
	R_{1} = \{&L_{1}:\{0,00,000\},L_{2}:\{1,11,111\},L_{3}:\{0,1\},L_{4}:\{\epsilon\},L_{5}:\{\}\}\\
		R_{2} = \{&\\
		&L_{1}:\{0,00,000\},L_{2}:\{1,11,111\},L_{3}:\{0,1\},L_{4}:\{\epsilon\},\\
		&L_{6}:\{0,00,000,1,11,111\}, L_{7}:\{0,00,000,1\},\\
		&L_{8}:\{0,00,000,\epsilon\}, L_{9}:\{0, 1, 11,111\}, L_{10}:\{1, 11,111,\epsilon\},\\
		&L_{11}:\{0,1,\epsilon\}\\
		\}&\\
		R_{3} = \{&\\
		&L_{6}:\{0,00,000,1,11,111\}, L_{7}:\{0,00,000,1\},\\
		&L_{8}:\{0,00,000,\epsilon\}, L_{9}:\{0, 1, 11,111\}, L_{10}:\{1, 11,111,\epsilon\},\\ &L_{11}:\{0,1,\epsilon\},L_{12}:\{0,00,000,1,11,111,\epsilon\},\\
		&L_{13}:\{0,00,000,1,\epsilon\},L_{14}:\{0,1,11,111,\epsilon\}\\
		&\}\\
		R_{4} =\{&\\
		&L_{6}:\{0,00,000,1,11,111\},L_{12}:\{0,00,000,1,11,111,\epsilon\},\\
	&L_{13}:\{0,00,000,1,\epsilon\},L_{14}:\{0,1,11,111,\epsilon\}\\
	&\}\\
	R_{5} = \{&L_{12}:\{0,00,000,1,11,111,\epsilon\}\}	\\
	R_{6} = \{&\}
	\end{split}
\end{equation*}
\section{Operace "rozdílné" (Operation Diferent)}
\label{section:OD}
Operation Diferent, nebo-li rozdílné (neplést s Diference, rozdíl), tato operace přijímá libovolné dva jazyky a provádí nad nimi operaci kterou nejlépe definuje \ref{eq:Different}, pokud bychom ji chtěli definovat do podrobna, byla by popsána rovnicí \ref{eq:Different_detail}

\begin{equation}\label{eq:Different}
Different(L_{1}, L_{2}) = Union(L_{1}, L_{2}) - Intersection(L_{1}, L_{2})
\end{equation}


\begin{equation}\label{eq:Different_detail}
Different(L_{1}, L_{2}) = \{w| (w\in L_{1}\land  w\notin L_{2}) \lor (w\notin L_{1}\land  w\in L_{2})\}
\end{equation}

\subsection{Vlastnosti}
Vzhledem k faktu, že tato operace jde rozložit na několik jednodušších operací, jsme tak schopni odvodit i chování této operace. Vzhledem k tomu, že všechny operace uvedené v \ref{eq:Different} jsou uzavřené nad regulárními jazyky, můžeme si říci, že celá operace je nad regulárními jazyky uzavřená. Stejným způsobem víme, že tato operace není uzavřená nad bezkontextovými jazyky.

Operace při opakovaném volání nevykazuje žádné speciální vlastnosti, vyjma toho, že jakékoliv použití této operace nad rodinou jazyků nám zaručí existenci prázdného jazyku. Existence prázdného jazyku nám dále zaručí, že v dalším výsledku nám nezmizí žádný jazyk a tedy je výsledná rodina vždy stejná, nebo větší.Po určitém počtu iterací se nakonec dostáváme k rodině, která je nad touto operací tranzitivně uzavřená

%\section{Reverzní konkatenace(Reverse Concatenation)}

\section{Unikátní konkatenace (Unique Concatenation)}
Unique Concatenation, nebo-li Výlučná konkatenace, je operace jenž se chová jako konkatenace dvou jazyků, až na ten fakt, že nepřijímá takové řetězce z K a L, jejichž konkatenace by patřila do K nebo L.
Pro zjednodušení si určeme konkatenaci, tak, že konkatenace je operace popsatelná rovnicí \ref{eq:Concatenation}.
Unikátní konkatenace poté je definována rovnicí \ref{eq:UniqueConcatenation_detail}.
Ve zkratce výsledek unikátní konkatenace tytéž řetězce jako konkatenace normální, s výjimkou, že neobsahuje řetězce které již byly obsaženy v jazicích jenž jsme konkatenovali, což je nejlépe popsáno rovnicí  \ref{eq:UniqueConcatenation}.

\begin{equation}\label{eq:Concatenation}
Concatenation(L_{1}, L_{2}) = \{w|w=xy; y\in L_{1}\land  y\notin L_{2}\}
\end{equation}

\begin{equation}\label{eq:UniqueConcatenation_detail}
UniqueConc(L_{1}, L_{2}) = \{w|w=xy; y\in L_{1}\land  y\notin L_{2} \land xy \notin L_{1} \land xy \notin L_{2} \}
\end{equation}

\begin{equation}\label{eq:UniqueConcatenation}
UniqueConc(L_{1}, L_{2}) = Concatenation(L_{1}, L_{2}) - L_{1} - L_{2}
\end{equation}

\subsection{Vlastnosti}
Díky smutné neunikátnosti této operace jsme schopni z jejího vyjádření \ref{eq:UniqueConcatenation} vyvodit, že bude uzavřená nad regulárními a neuzavřená nad bez kontextovými jazyky.

Rodina vzniklá touto operací je velikosti $n^2$ nebo $(n-1)^2+1$ pokud původní rodina obsahovala prázdný jazyk. 

\section{Prefixes}
Prefixes je operace, jejíž aplikace na K vytváří L, kde L obsahuje všechny řetězce, které jsou prefixy všech řetězců K. (Viz. \ref{eq:Prefixes})

\begin{equation}\label{eq:Prefixes}
Prefixes(L) = \{w|x=wy; x \in L \land y \in \Sigma_{L}\}
\end{equation}

\subsection{Vlastnosti}
\begin{theorem}[Uzavřenost nad regulárními jazyky] \label{thm:PrefixReg}
	Použití operace Prefixes nad libovolným regulárním jazykem $L$ generuje opět regulární jazyk.
\end{theorem}

\begin{proof}[Důkaz \ref{thm:PrefixReg}]
	Mějme konečný automat $M=\{Q_{M}, \Sigma_{M}, R_{M}, s_{M}, F_{M}\}$ pokud je operace Prefixes uzavřená nad regulárními jazyky, je tedy uzavřená nad konečnými automaty a tedy jsme schopni vytvořit konečný automat $N$, který bude přijímat všechny prefixy jazyka $L(M)$.
	
	Automat $N$ vytvoříme tak, že vezmeme automat $M$ a všechny stavy, které které nejsou neukončující označíme jako konečné.
	
	Automat $N$ je vždy Konečný Automat a přijímá všechny prefixy jazyka L($M$), což potvrzuje, že je operace nad regulárními jazyky uzavřená.
\end{proof}

\subsection{Příklad}
Nechť existuje jazyk $L=\{123, 456, 101\}$, Aplikací operace Prefixes můžeme tedy vytvořit jazyk $K=Prefixes(L) = \{'', 1, 12, 123, 4, 45, 456, 10, 101\}$. Zde je třeba si všimnout faktu že řetězec "123" je prefixem řetězce "123", neboť sufixem je ''

\section{Shuffle}
Operaci shuffle můžeme definovat jako všechny kombinace které vzniknou pomícháním znaků z řetězců u a v se zachováním pořadí znaků tak jak byly v původním řetězci. V podstatě bychom tuto operaci mohli popsat jako prolnutí dvou řetězců. (viz \ref{eq:Shuffle}) Tuto operaci potom můžeme generalizovat tak na jazyky, viz \ref{eq:ShuffleLang}. Znalost a pochopení této operace nám pomůže v chápání dalších operací, jako je vkládání(Kapitoly \ref{sec:seqIns} a \ref{sec:parIns}).


\begin{equation}\label{eq:Shuffle}
\begin{split}
Shuffle(u,v) = \{& \\
& u_{1}v_{1}...u_{i}v_{j}|\\
&\tab u=u_{1}...u_{i} \land u \in \Sigma* \\
&\tab v=v_{1}...v_{j} \land v \in \Sigma*\\
&\tab u_{p} \in u \land u_{p} \in \Sigma \cup \epsilon;\\
&\tab v_{q} \in v  \land v_{q} \in \Sigma \cup \epsilon;\\
&\tab 1 \leq p \leq i \land 1 \leq q \leq j\\
\}&
\end{split}
\end{equation}

\begin{equation}\label{eq:ShuffleLang}
Shuffle(K,L) = \{w|w=Shuffle(u,v); u \in K \land v \in L \}
\end{equation}


\subsection{Vlastnosti}
Operace shuffle je uzavřená nad regulárními jazyky a neuzavřená nad jazyky bezkontextovými.

\subsection{Příklad}
\textbf{Operace Shuffle nad dvěma řetězci}:\\
Nechť existují řetězce $u$ a $v$, ta že $u = "ab", v = "cd"$ 
$Shuffle(u,v)$ se poté bude rovnat $\{"abcd", "acbd", "acdb", "cabd", "cadb", "cdab"\}$

\section{Sekvenční vkládání (Sequential Insertion)}\label{sec:seqIns}
Tato operace je krásně popsána v \cite{lilaKari} strana 23-28. Ale ve svém provedení jako takovém je se jedná o jednoduchou operaci, kdy do libovolného řetězce z jazyka $K$ vložíme na libovolné místo libovolný řetězec z jazyka $L$.
Tuto operaci si tedy můžeme popsat rovnicí \ref{eq:seqIns}.
\begin{equation}\label{eq:seqIns}
	SequentialInserion(K,L) = \{w|w=xyz, xz \in K \land x \in L\}
\end{equation}

\subsection{Vlastnosti}
Operace je uzavřená nad regulárními a bezkontextovými jazyky a samozřejmě není komutativní, viz \cite{lilaKari} strana 25-27.
\subsection{Příklad}
\textbf{Operace SequentialInsertion nad dvěma jazyky:}\\
Mějme dva jazyky, $K={abc,def}$ a $L={xy}$\\
Použití operace nám poté generuje jazyk:\\ $SequentialInsertion(K,L) = \{xyabc,axybc,abxyc,abcxy,xydef,dxyef,dexyf,defxy\}$

\section{Paralelní vkládání (Parallel Insertion)}\label{sec:parIns}
Paralelní vkládání je operace velice podobná sekvenčnímu, avšak s tím rozdílem, že pokud máme jazyk $K$ do kterého vkládáme, tak nevkládáme pouze na jednu libovolnou pozici, nýbrž vkládáme na všechny pozice. Operace je opět krásně popsána v \cite{lilaKari} na straně 24-28. Tuto operaci bychom si také mohli definovat rovnicí \ref{eq:parIns}

\begin{equation}\label{eq:parIns}
	\begin{split}
	ParallelInsertion(K,L) =\{& \\
	&w|w=x_{0}u_{0}u_{1}x_{1}...x_{n}u_{n}x_{n+1},\\
	&\quad x_{k} \in L \land u_{1}u_{1}...u_{n} \in K,\\
	&\quad n \geq 0	\land 0\leq k \leq n\\	
	\}&
	\end{split}
\end{equation}

\subsection{Vlastnosti}
Paralelní vkládání je uzavřené nad rekulárními a bezkotextovými jazyky, viz \cite{lilaKari} na straně 27-28.

\subsection{Příklad}
\textbf{Operace ParallelInsetion nad jazyky:}\\
Mějme jazyky $K=\{abc\}$ a $L=\{de\}$.\\
Použití operace nám poté generuje jazyk: $ParallelInsertion(K,L) = \{de\textbf{a}de\textbf{b}de\textbf{c}\}$
\section{Protkání (Interlacement)}
Tato operace je inspirovaná Shuffle a vkládajícími operacemi. Tato operace je nejvíce podobná operaci Shuffle, kdy nejdříve uvažujme dva řetězce $u$ a $v$. Protkáním těchto dvou řetězců dostáváme řetězec, který se skládá vždy ze symbolu řetězce $u$ a následně symbolu z řetězce $v$. Protkání dvou řetězců si tedy můžeme definovat rovnicí \ref{eq:interlaceStr} s tím, že pokud dva řetězce nemají stejnou délku, nemohou být protkány, jinak se generuje řetězec prázdné délky. Tuto operaci poté následně můžeme generalizovat a použít nad jazyky obdobným způsobem jako jakoukoliv jinou operaci, tedy protkáme každý řetězec z jazyka $K$ s každým řetězcem z jazyka $L$ (viz. rovnice \ref{eq:interlaceLang}). Obdobně bychom mohli operaci generalizovat i na rodinu jazyků (viz rovnice \ref{eq:interlaceFam}). Iterativní použití této operace nad rodinou jazyků by mohlo mít silně expandující efekt, proto si představme ještě variaci, která nám nebude přijímat dva stejné jazyky, tedy může bude růst výrazně pomaleji, případně může za správných okolností stagnovat, i klesat \ref{eq:uniqueInterlaceFam}

\begin{equation}\label{eq:interlaceStr}
	\begin{split}
	Interlacement(u,v) = \{&\\
		&w|w=u_{1}v_{1}u_{2}v_{2}...u_{k}v_{k};\\
		&u=u_{1}u_{2}...u_{k} \land v=v_{1}v_{2}...v_{k};\\
		&k \geq 1 \land k= |u| = |v|\\
	\}&
	\end{split}
\end{equation}

\begin{equation}\label{eq:interlaceLang}
	Interlacement(K,L) = \{w|w=Interlacement(u,v); u\in K \land v \in L; w \neq \epsilon\}
\end{equation}


\begin{equation}\label{eq:interlaceFam}
	Interlacement(R) = \{J|J=Interlacement(K,L); K,L\in R; J\neq \emptyset\}
\end{equation}


\begin{equation}\label{eq:uniqueInterlaceFam}
UniqueInterlacement(R) = \{J|J=Interlacement(K,L); K,L\in R \land K \neq L;J \neq \emptyset\}
\end{equation}


\subsection{Vlastnosti}
\begin{theorem}[Uzavřenost nad regulárními jazyky] \label{thm:InterlacementReg}
	Libovolné dva regulární jazyky, jsou uzavřené nad operací Interlacement.
\end{theorem}

\begin{proof}[Důkaz \ref{thm:InterlacementReg}]
	Mějme automaty $M=\{Q_{M},\Sigma_{M}, R_{M}, s_{M}, F_{M}\}$ a $N=\{Q_{N},\Sigma_{N}, R_{N}, s_{N}, F_{N}\}$ a jimi definované jazyky $K(M)$ a $L(N)$.
	Nyní jsme schopni vytvořit jazyk $K^{'}(M^{'})=Interlacement(K,L)$:
	\begin{equation}\label{eq:interlaceProof}
		\begin{split}
			M^{'} = \{&\\
			&Q = \{0,1\}\times Q_{M}\times Q_{N},\\
			&\Sigma = \Sigma_{M} \cup \Sigma{N},\\
			& R = \{\\
			&\quad	(0,q_{M},q_{N})\alpha \longrightarrow(1,q_{M}\alpha,q_{N}),\\
			&\quad	(1,q_{M},q_{N})\alpha \longrightarrow(0,q_{M},q_{N}\alpha);\\
			&\quad q_{M} \in M \land q_{N} \in N \\
			&\}\\
			&s = (0,s_{M},s_{N}),\\
			&F = \{(0,q_{M},q_{N})| q_{M} \in F_{M} \land q_{N} \in F_{N}\}\\
			\}&
		\end{split}
	\end{equation}
	A jelikož jsme schopni sestrojit konečný automat přijímající tento jazyk, víme, že tento jazyk bude regulární. Tedy dostáváme vztah:
	$$Interlacement(K,L)=K^{'}(M^{'})$$ 
\end{proof}
\bigskip 
Na první pohled by se mohlo zdát, že konečný automat pro tuto výsledek této operace sestrojit nedokážeme, neb se nám sama nabízí možnost zásobníkového automatu, který by "počítal" jestli bychom měli aplikovat pravidla z prvního nebo druhého automatu, ale jelikož zásobník takového automatu by vždy obsahoval jeden symbol (vyjma počátečního), tak jsme toto chování schopni replikovat nad konečnými automaty pomocí zdvojnásobení počtu stavů.

Použití této operace nad jazyky, může generovat prázdný jazyk, z čehož je zřejmé, že né všechny řetězce přispějí do výsledného jazyka, přesněji řetězec z jazyka $K$ takové délky, že neexistuje řetězec v jazyce $L$ stejné délky, nepřispívá do generovaného jazyka a naopak.

Použití operace nad dvěma jazyky vždy generuje jazyk nepřijímají prázdný řetězec.

Použití operace neprázdnou rodinou jazyků vždy generuje větší, nebo alespoň stejně velkou rodinu jazyků, s tím že při iterovaném použití, se výsledek nikdy neustálí a tedy není možné aby byla rodina jazyků nad touto operací tranzitivně uzavřena. 
  
  
Pokud bychom uvažovali upravenou verzi, nepřijímající dva stejné jazyky (Jak jsou dva jazyky porovnávány jsme si ukazovali v kapitole \ref{se:DiferentUnion}), tak budeme li uvažovat rodinu jazyků $R$ dostáváme následující vlastnosti:
\begin{enumerate}
	\item Použití operace nad libovolnými dvěma nestejnými jazyky z $R$ vrací prázdný jazyk:
	
	Generovaná rodina je prázdná
	
	\item \label{enum:interlace1} Obsahuje-li $R$ pouze dva nestejné jazyky, nad nimiž operace negeneruje prázdný jazyk:
	
	Generovaná rodina bude obsahovat opět dva stejně velké jazyky jazyky ($Interlacement(K,L) a Interlacement(L,K)$)
	
	\begin{enumerate}
		\item Při iterovaném použití nad takovou rodinou:
		
		Od druhého použití víme, že rodina $R_{i}$ bude stejně velká jako rodina $R_{i-1}$, avšak jazyky v rodině $R_{i}$ budou větší než v rodině $R_{i-1}$ a to samé platí i o délce nejdelšího řetězce. 
	\end{enumerate}

	\item Obsahuje li rodina $R$ více kompatibilních jazyků, víme pouze, že při iterativním použití bude od kroku 2 růst, s tou výjimkou, že obsahuje li rodina pouze dvojce kompatibilních jazyků, bude stagnovat obdobně jako v bodě \ref{enum:interlace1}
\end{enumerate}
  
\subsection{Příklad}
\textbf{Použití nad dvěma řetězci:}\\
Nechť existuje řetězec $u=abc$ a řetězec $v=def$, poté řetězce $w=Interlacement(u,v)$ a $w'=Interlacement(v,u)$ jsou:\\
$w=adbecf$ a $w'=daebfc$

\textbf{Použití nad dvěma Jazyky:}\\
Nechť existují jazyky $K=\{01,23,45\}$ a $L=\{67, 890\}$, poté jazyky $K^{'}=Interlacement(K,L)$ a $L^{'}=Interlacement(L,K)$ jsou:\\
$K^{'}=\{0617,2637,4657\}$ a $L^{'}=\{6071,6273,6475\}$

\textbf{Použití nad rodinou jazyků:}\\
Nechť existuje rodina jazyků $R=\{L_{1}:\{01,23,45\}, L_{2}:\{67,890\}, L_{3}:\{89,ab\}\}$ poté použití operace nad touto rodinou bude vypadat následovně:
\begin{equation*}
	\begin{split}
	R^{'}=Interlacement(R)=\{&L_{1,1}:\{0011,0213,0415,2031,2233,2435,4051,4253,4455\},\\&L_{1,2}:{0617,2637,4657},L_{2,1}:\{6071,6273,6475\},...,\\&L_{3,2}:\{5697,a6b7\},L_{3,3}:\{8899,aabb,8a9b,a8b9\}\}
	\end{split}
\end{equation*}

\textbf{Použití unikátní verze operace nad rodinou jazyků:}\\
Nechť existuje rodina jazyků $R=\{L_{1}:\{01,23,45\}, L_{2}:\{67,890\}, L_{3}:\{89,ab\}\}$ poté použití unikátní verze operace nad touto rodinou bude vypadat následovně:
\begin{equation*}
\begin{split}
R^{'}=UniqueInterlacement(R)=\{&L_{1,2}:\{0617,2637,4657\},L_{2,1}:\{6071,6273,6475\},\\&L_{1,3}:\{0819,2839,4859,0a1b,2a3b,4a5b\},...,\\&\{L_{3,2}:\{5697,a6b7\}\}
\end{split}
\end{equation*}



\section{Sekvenční Mazání (Sequential Deletion)}
Sekvenční mazání je operace vzdáleně podobná sekvenčnímu vkládání a to tak, že bychom ji mohly nazvat přesným opakem. Takto si ji však můžeme nazvat pouze neformálně, nemůžeme u této operace spoléhat na $SequentialDeletion(SequentialInsertion(L))=L$(viz příklad).
Více je tato operace popsána v \cite{lilaKari} na straně 55-70. Tato operace je lehce popsatelná rovnicí \ref{eq:seqDel}.

\begin{equation}\label{eq:seqDel}
	SequentialDeletion(K,L) = \{w|w=xz; xyz \in K \land y \in L\} 
\end{equation}

\subsection{Vlastnosti}
Všechny vlastnosti si dopodrobna může čtenář přečíst ve výše uvedené knize, jako sumarizaci se zde však hodí podotknouti, že operace je uzavřená nad regulárními jazyky.  

\subsection{Příklad}
\textbf{Ukázka toho, že $SequentialDeletion(SequentialInsertion(K,L),L)\neq K$:}\\
Mějme jazyky $K=\{abc\}$ a $L=\{a\}$. Použijeme-li sekvenční vkládání, dostáváme $K_{2} = SequentialInsertion(K,L)=\{aabc,abac,abca\}$. Pokud následně použijeme sekvenční mazání, dostáváme $K_{3}=SequentialDeletion(K_{2},L) = \{abc,bac,bca\}$ což se zcela zřetelně nerovná $K$.

%\section{Paralel deletion}
Paralelní mazání je jedna z mnoha operací soustředících se na mazání jednoho řetězce z řetězce druhého. Narozdíl od Sequential deletion (viz TODO reference), která maže výskyt řetězce u v řetězci v avšak s tím rozdílem, že maže všechny nepřekrývající se výskyty řetězce u. (viz [todo dodělat roovnici, je definována v http://www.fit.vutbr.cz/research/pubs/TR/2005/sem\_uifs/s050121clanek.pdf, ale tomu nerozumim => pochopit => zjednodušit na 2 řetězce, bez použití jazyku => zobecnit na použití nad rodinou jazyků]) 

\subsection{Vlastnosti}
TODO: vlastnosti při iterovaném použití
Vlastnosti nad regulárními, bezkontextovými jazyky
vlastnosti nad jazyky obsahující řetězce nakonečné délky

\section{Operation SubAlpha}
Toto je jen pomocná operace, kterou si definujeme písemně pro další použití. Operace SubAlpha vrací všechny znaky které řetězec, nebo jazyk obsahuje. (Viz příklad)

\subsection{Příklad}
$$\Sigma_{1} = \{"a", "b", ..., "z"\}, L = \{"abc", "ade"\}$$

z čehož plyne 

$$u,v \in \Sigma_{1}$$. 

Provedeme-li nad L operaci SubAlpha získáme: 

$$\Sigma_{2} = SubAlpha(L) = \{"a", "b", "c", "d", "e"\}$$

a můžeme též tvrdit 

$$Sigma_{2} \subseteq Sigma_{1}$$


\section{Plné zakázání abecedy (Full Alphabet deletion)}
Tato operace je velmi výrazně inspirovaná Paralelním mazáním \cite{lilaKari} strany 55-70, a je zcela aplikovatelná použitím Paralelního mazání, kde bychom místo abecedy použily jazyk, kde každý řetězec je výlučně jeden symbol. Rozdíl této operace s je však v její aplikovatelnosti na konečné automaty (viz \ref{subsec:fadProp}), případně v její implementovatelnosti [TODO reference]. Tato operace je opět lehce popsatelná rovnicí

\begin{equation}
	\begin{split}
		FAD(K,\Sigma_{2}) = \{&\\
		&u_{1}u_{2}...u_{k}u_{k+1}|k\geq1, u_{i} \in \Sigma^{*}, 1\leq i \leq k+1 \land\\
		&\exists v_{i} \in \Sigma_{2}, 1 \leq i \leq k:u=u_{1}v_{1}...u_{k}v_{k}u_{k+1}v_{k+1}\\
		&kde \{u_{i}\} \notin \Sigma_{2}\\
		\}&
	\end{split}
\end{equation}

\subsection{Vlastnosti}\label{subsec:fadProp}
Vzhledem k tomu, že tato operace se dá aplikovat pomocí paralelního mazání, tak víme že stejně jako paralelní mazání bude uzavřená nad regulárními jazyky. Výhodou této operace je však to, že oproti paralelnímu mazání, je ta, že není ani zdaleka tak složitý. Paralelní mazaní samo o sobě vyžaduje velkou režii, a aplikovatelnost nad automatem je velice složitá, kdežto u Plného mazání abecedy, je aplikace primitivní, neb nám pouze stačí projít všechna pravidla automatu ze kterého odstraňujeme abecedu a nahradit odstraňované symboly za $\epsilon$.

\subsection{Příklad}
 
%\input{operace/mix.tex}
\section{Pop}
Operace Pop, de facto operace reverzní ke konkatenaci. Tuto operaci si rozdělíme na $LPop(K,L)$ a $RPop(K,L)$. Toto rozdělení provádíme z toho důvodu, že konkatenovat můžeme z obou stran, a tedy bychom rádi z obou stran i odebírali, což musíme nějak specifikovat. Obě operace pop jsou každá, vlastním způsobem omezená operace Sekvenční mazání.
Obě operace si můžeme definovat následujícím způsobem:

\begin{equation}\label{eq:lPop}
	LPop(K,L) = \{y|xy |in K \land x \in L\}
\end{equation}


\begin{equation}\label{eq:rPop}
	RPop(K,L) = \{x|xy |in K \land x \in L\}
\end{equation}

\subsection{Vlastnosti}
\begin{theorem}[Uzavřenost lPop nad regulárními jazyky] \label{thm:lPopReg}
	Libovolné dva regulární jazyky, jsou uzavřené nad operací lPop.
\end{theorem}

\begin{proof}[Důkaz \ref{thm:lPopReg}]
	Nechť existují dva jazyky $L_{1}$ a $L_{2}$, nad abecedou $\Sigma$ a nechť existuje automat $M=\{Q,\Sigma,R,s,F\}$ přijímající jazyk $L_{1}$. Pro každé dva stavy $s,q \in Q$  nechť existuje:
	$$L_{s,q} = \{w|sw\longrightarrow^{*}q \in M; w\in \Sigma\}$$
	 Poté uvažujme automat:
	 $$M^{'}=\{Q,\Sigma\cup\{\#\},R^{'},s,F\}$$
	 kde $$R^{'} = R\cup\{s\#\longleftarrow q|s,q\in Q \land L_{2} \cap L_{s,q} \neq \emptyset\}$$
	 kde $\#$ je nový symbol nepatřící do abecedy $\Sigma$.
	 
	 Následně můžeme tvrdit že:
	 $$LPop(L_{1},L_{2}) = h(L(M^{'} \cap \Sigma^{*}\#\Sigma^{*}))$$
	 
	 Následně důkaz pokračuje, jako důkaz Sekvenčního Mazání, viz \cite{lilaKari} strany 60-61, kde $A^{'}=M^{'}$.
\end{proof}

\begin{theorem}[Uzavřenost rPop nad regulárními jazyky] \label{thm:rPopReg}
	Libovolné dva regulární jazyky, jsou uzavřené nad operací rPop.
\end{theorem}

\begin{proof}[Důkaz \ref{thm:lPopReg}]
	Nechť existují dva jazyky $L_{1}$ a $L_{2}$, nad abecedou $\Sigma$ a nechť existuje automat $M=\{Q,\Sigma,R,s,F\}$ přijímající jazyk $L_{1}$. Pro každé dva stavy $q \in Q \land f \in F$  nechť existuje:
	$$L_{q,f} = \{w|qw\longrightarrow^{*}f \in M; w\in \Sigma\}$$
	Poté uvažujme automat:
	$$M^{'}=\{Q,\Sigma\cup\{\#\},R^{'},s,F\}$$
	kde $$R^{'} = R\cup\{s\#\longleftarrow q|s,q\in Q \land L_{2} \cap L_{q,f}\}$$
	kde $\#$ je nový symbol nepatřící do abecedy $\Sigma$.
	
	Následně můžeme tvrdit že:
	$$LPop(L_{1},L_{2}) = h(L(M^{'} \cap \Sigma^{*}\#\Sigma^{*}))$$
	
	Následně důkaz pokračuje, jako důkaz Sekvenčního Mazání, viz \cite{lilaKari} strany 60-61, kde $A^{'}=M^{'}$.
\end{proof}

Při používání této operace, je důležité mít na paměti, že i-když o ní můžeme uvažovat jako o opaku konkatenace, tedy $RPop(Concatenation(K,L),L) = K$, tak $Concatenation(RPop(K,L),L) \neq K$. A to protože, do $Pop$ operací nepřispívají všechny řetězce jazyka K, nýbrž pouze ty, ze kterých můžeme něco smazat.

\subsection{Příklad}
\textbf{Ukázka použití LPop a RPop:}\\
Nechť existují dva jazyky, $K=\{aba,ab,c\}$ a $L=\{a\}$.

Použití operace $LPop$ nám generuje jazyk $K^{'}=LPop(K,L) =\{ba,b\}$ a použití operace $RPop$ nám generuje jazyk $K^{''}=RPop(K,L) =\{ab\}$

\textbf{Ukázka $\mathbf{RPop(Concatenation(K,L),L) = K}$:}\\
Uvažujme stejné jazyky jako v předchozím příklade, konkatenace $Concatenation(K,L)$ nám generuje jazyk $K^{'}=\{abaa,aba,ca\}$. Následné použití RPop nám generuje opět jazyk $K$, $RPop(K^{'}, L) = K$.

\textbf{Ukázka $\mathbf{Concatenation(RPop(K,L),L) \neq K}$:}\\
Opět uvažujme stejné jazyky jako v předchozích příkladech, použití operace $RPop(K,L)$ nám generuje jazyk $K^{'}={ab}$, což se zcela zřetelně nerovná $K$











% \section{Operation Len}
% Tato operace je další pomocná operace. Aplikace této operace na řetězec vrací délku řetězce.
% \section{Permuted Scattered Insertion}
% TODO (přejmenovat operaci, špatně jsem ji pochopil,u nemá být přeházené [ale nemazat, správné PSI dopsat když bue čas a místo])
% Tato operace je velmi podobná Shuffle, s tou výjimkou, že nevyžaduje zachování pořadí. (viz. \ref{eq:PSI} [TODO něco mi tam chybí, pravděpodobně nějak vyjádřit, že každá pozice R je použita jen jednou])
% \begin{equation}\label{eq:PSI}
% \begin{split}
% PSI(u,v) = \{& \\
% & u_{1}v_{1}...u_{i}v_{j}|\\
% &\tab u=u_{r_{1}}...u_{r_{i}} \land u \in \Sigma* \\
% &\tab v=v_{s_{1}}...v_{s_{j}} \land v \in \Sigma*\\
% &\tab u_{p} \in u \land u_{p} \in \Sigma \cup \epsilon;\\
% &\tab v_{q} \in v  \land v_{q} \in \Sigma \cup \epsilon;\\
% &\tab 1 \leq p \leq i \land 1 \leq q \leq j;\\
% &\tab 1 \leq r \leq i \land 1 \leq s \leq j;\\
% \}&
% \end{split}

% \end{equation}

% \subsection{Vlastnosti}
% TODO (zatím odhady)
% Zajímavou vlastností této operace je ta, že jí můžeme zjednodušit na \ref{eq:PSI_p1}. Pokud nad tím uvažujeme takto, tak nám je zřejmé, že použitím operace na jazyk nám vzniká jazyk nový, který je možno popsat pomocí JA ("Jumping automata"). [Pozn.: tohle si ověřit, je to jen domněnka a nevím jaký bude mít použití této operace reálný dopad na vzniklý jazyk, protože ten JA by se musel hodně omezit, vhledem k faktu, že některé přijímané řetězce budou mít omezenou délku, takže mi vzniká jazyk, který bude obsahovat ab,ba, abcd...dcab, ale nebude obsahovat abc ]
% \begin{equation}\label{eq:PSI_p1}
%     PSI(u,v) = \{w| w = \Sigma_{2}^n; \Sigma_{2} = (SubAlpha(u) \cup SubAlpha(v)) \land n = (Len(u)+Len(v)); u,v \in \Sigma_{1}\}
% \end{equation}

